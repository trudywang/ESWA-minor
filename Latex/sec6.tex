\section{Conclusions}
\label{sec:con}
We address the problem of refreshing cache contents in a changed network.
Changes of road conditions are commonly seen in the real world, i.e., traffic congestion or road constructions.
Shortest path caches need refreshing periodically, so that the information is accurate and valid and the cache utilization is maximized.
To the best of our knowledge, our work is the first one to discuss the cache refreshment problem when an edge changes its weight.
In order to store shortest paths and answer queries efficiently, a bitmap-based cache structure is used.
We introduce the concept of affected paths, and then exploit the properties associated with affected paths.
In the following, we develop algorithms to detect affected shortest paths resulted from road network changes.
Sequentially, four cache refreshment strategies are illustrated.

A series of experiments are conducted. The performance of cache structure and affected path detection methods are considerable.
Computational results showcase that RLB is as good as RWC in terms of hit ratio while the former has a much shorter refreshment time.
Our work has many applications, such as refreshment of shortest paths caches of map companies.
We suggest that RLB is used to refresh caches at short intervals and RWC is used at long intervals. How to trade off the use frequency of two methods is based on the real applications.

The weakness of our work, even of most works of this kind, is that we rely heavily on the information of query logs, although we introduce roulette wheel selection to avoid over fitting.
It is clearly that future queries are not a precise copy of query logs. Hence, the hit ratio is to some extent restricted by query logs.
In the future, works may focus on the following directions. The cache structure, the method to detect affected paths and the strategies to refresh caches can be studied solely and then consolidated, obtaining a strengthened cache refreshment solution. The problem of ``batch of updates'' should be considered. In this paper, we only consider the change of single edge. The story of batch changes is not a simple repetition of single change, since edges could interfere. The impact of multiple edges is combinatorial.
Currently we just consider the changes of the graph; it would be interesting to investigate the changes of query patterns. Of course, it may be possible to study directed networks in the future as in the real life one direction of a road may be closed for construction or other purposes while the other direction is still open.

\subsubsection*{Acknowledgments.}
{The work is partially supported by the National Basic Research Program of China (973 Program) (No. 2012CB316201), the National Natural Science Foundation of China (No. 61322208, 61272178, 61129002), the Doctoral Fund of Ministry of Education of China (No. 20110042110028), the Fundamental Research Funds for the Central Universities (No. N120504001, N110804002), and the Support Plan for Young Teachers of Shanghai (No. ZZSD15095).}


