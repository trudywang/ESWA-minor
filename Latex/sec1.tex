\section{Introduction}
Shortest path query systems are now widely seen in our daily life \citep{wu2012shortest,potamias2009fast,wei2010tedi,liu2011generalization,
cheng2012efficient}.
People who use Google or Baidu map usually request shortest paths to restaurants, parking lots, cinemas, playgrounds, shopping malls and so on.
When a user has a shortest path query, a cache in a local server is accessed; the result, if exists, will be directly returned to the user. If the cache does not have an off-the-shelf result to the query, the system accesses a global server which runs a shortest path algorithm for the result. The latter situation undoubtedly costs communication and computational time \citep{altingovde2009cost,baeza2007impact,Kriegel2008Hierarchical,gan2009improved,markatos2001caching}. The contents in caches are, therefore, very critical for the efficiency of the whole query system.

In the shortest path caching problem, existing works all discuss the cache initialization problem which addresses how to load an empty cache when a road network and a query log is known, so that the cache works in a high-efficiency level in the future. Researchers usually select paths with the highest query frequencies into caches, or those paths that contain the most numbers of nodes~\citep{thomsen2012effective,li2013}.
Existing works do not discuss how to refresh the cache when the road condition or queries change.
Actually, the road condition cannot remain the same all the time; on contrast, it changes frequently.
For example, rush hours or traffic congestion changes weights of some edges on a road network.
%\modify{This kind of network is called \emph{dynamic road network}(\cite{lee2007,tian2009monitoring,d2014fully}).}
When a road network changes, if paths in the cache are not adjusted, some paths in the cache may become invalid or the utilization of the cache decreases.
%\modify{In this paper, we call this kind of invalid shortest path, caused by the cost change of network, as \emph{affected shortest path}.}
How to maintain a cache based on changes of a road network is seldom discussed in the literature.
Apparently, a straightforward way to refresh a cache is to empty the cache and conduct an initialization process once more. Such a method is too naive, because computing and evaluating all the shortest paths once more are time-consuming.

In this paper, we discuss how to refresh a cache when the weight of a certain edge changes.
Assuming that only one edge changes its weight is reasonable.
On one hand, studying the change of single edge is a good point to start from.
We can draw some conclusions from the single-edge scenario and apply them to multi-edge scenario.
On the other hand, single-edge has some applications. For example, traffic congestion may happen in main streets and these main streets are far away from each other and unrelated; the congestion in every main street can be separately viewed as a single-edge scenario.

We first develop a cache structure based on a structure of bitmap to store and retrieve shortest paths. We then introduce the concept of ``affected paths" and develop algorithms to detect affected shortest paths due to the change of edge weight.
Next, we design four strategies to refresh the cache and compare their performance.
Our work contributes to the literature in two aspects. First, to the best of our knowledge, our work is the first one to discuss the shortest path caching problem in a changing graph.
This problem is undoubtedly more close to real applications.
Second, four cache refreshment strategies are proposed to update the cache. It is noteworthy that this work is an extension of \cite{Li2014refreshment} which is published in a conference.
This paper extends the initial work, by 1) adopting a bitmap index to efficiently answer queries of shortest paths; 2) introducing several optimization techniques to further improve the efficiency of detecting affected shortest paths; 3) performing more comprehensive experiments on real data sets, including Aalborg, Beijing and Singapore road networks. Especially, the data set of Singapore is newly added and is not used in the conference version.

%\modify{
%The shortest path cache refreshment problem we have studied in this paper falls into the scope of intelligent systems. The methodologies and strategies used, such as randomness, wheel of fortune, frequency first strategy, also belong to intelligent systems where can load cache in an intelligence manner and thus help improve efficiency of computing query.
%}

The rest of the paper is organized as follows. In Section~\ref{sec:rw}, we review works related to the shortest path caching problem.
%In Section~\ref{sec:pre}, we introduce all the concepts and formally define the problem.
In Section~\ref{sec:Cache Structure}, we formally define the problem and design a bitmap-based cache structure to store shortest paths.
Then we explore the properties of changed graphs and present our algorithm of detecting affected shortest paths in Section~\ref{sec:detect-affected}.
In the following, four cache refreshment strategies are proposed in Section~\ref{sec:strategies}.
In Section~\ref{sec:exp}, we conduct experimental comparisons of the proposed methods on real data sets.
Finally, we give closing remarks in Section~\ref{sec:con}.


\section{Literature Review}
\label{sec:rw}
%\fixme{we should give the notion of affected shortest path in introduction section.}
\modify{
Our problem is related to works of two streams. One stream is to detect which shortest paths are changed after the graph changes. We call this kind of problem as ``affected shortest path detection'' problem.
The other stream addresses the problem of caching shortest paths.
In the following, we introduce concrete works in the two streams.
}
\modify{
\subsection{Affected Shortest Path Detection}
}
%\fixme{Should we use dynamic or changed to describe graph?}
\modify{
The shortest path of a certain node-pair may change after the weights of some edges change in a graph. 
The affected shortest path detection problem deals with finding node-pairs whose shortest paths are affected and update new shortest paths for these node-pairs.
Based on whether the network changes in a regular manner, it has predictable and unpredictable networks.
}

\modify{
\textbf{Predictable Network Models} aim at finding the expected fastest (shortest) paths based on the known distribution of predicted networks. In details, \cite{kanoulas2006finding} developed the concept of speed pattern to specify the predicted network costs (weights) at different time, and the fastest paths corresponding to different time can be computed based on the predicted network cost. \cite{fu1998expected} modeled the cost of each edge as a continuous stochastic process, and then they found the expected fastest paths. Similarly, \cite{ding2008finding} used time-delay functions to represent the predicted edge costs.
%However, in our scenario, we consider the dynamic road networks as unpredictable, the edge cost is assumed to change quickly and then remain invariable until next changes. The next related works we introduced are based on the unpredictable dynamic network.
}


\textbf{Unpredictable Network Models} assume that the costs of graph change randomly and cannot be depicted by a statistical distribution. Unpredictable network problem has offline and online versions. 
%In this paper, we refer to the offline version as pre-processing and storing additional data except the shortest path of queries\fixme{please note this sentence, and polish it}.

For the offline research, it has pre-processing and needs extra space to store auxiliary information.
In \cite{tian2009monitoring}, an index named QSI was used to identify the shortest paths affected by the change of network. 
QSI keeps the subnetworks that are involved in derivation of answering shortest paths. 
In \cite{lee2007}, in order to process more than one shortest path query simultaneously, a grid-based index structure was designed to record all the affecting areas.
Besides, arc-flags is a typical index-based approach to compute the shortest paths. It has been adopted by \cite{lauther2004extremely} for a static network. \cite{berrettini2009arc,d2014fully} studied the problem of updating arc-flags in dynamic networks.

For the online research, \cite{bauer2009batch,d2013dynamically,frigioni1996fully,narvaez2000new,mcquillan1980new} studied how to update the shortest paths starting from a specified node.
In general, these algorithms preprocess the shortest paths starting from a given source node, and store them in a shortest path tree (SPT). To update the SPT, the basic idea is to utilize the information of the outdated SPT and update only the part of the SPT that is affected by the changed edge.
\cite{lee2007} studied the problem of continuous evaluating shortest path queries. For several queries, they developed an ellipse bound method (EBM) where each shortest path corresponds to an elliptic geographical area, called affected area, and all the updated edges are considered to affect their corresponding paths.

It can be seen from the above review, offline study of shortest path detection problem uses extra space. For the online problem, some research maintain a novel designed data structure or they can only detect known paths. In our problem, the space is reserved for the cache; it cannot tell which paths are affected (or no longer the shortest) before the graph changes. Thus, extant algorithms for detecting affected shortest paths cannot be used in our problem directly.


%\iffalse
%\textbf{Continuous Evaluation of Shortest Path Query}. Continuous evaluation of shortest path query has been studied in \cite{lee2007,tian2009monitoring}, which actually solves the problem of detecting affected shortest paths in dynamic road networks. \cite{lee2007} proposes an ellipse bound method (EBM), where each shortest path corresponds to an elliptic geographical area, called affected area, and all the updated edges are considered to affect their corresponding paths. EBM can solve this problem when source and destination nodes are given in an online fashion. However, for our scenario, it needs to perform the algorithm for each shortest path in cache, which will result in costly computations. Although \cite{lee2007} propose a method to process more than on continuous shortest path query simultaneously, a grid-based index structure is needed to record all the affecting areas. In \cite{tian2009monitoring}, an index, named QSI, is needed to identify the shortest paths affected by the network changes. Both the above methods require extra storages, which is infeasible in our scenario.
%
%
%\modify{
%\textbf{Single Source Shortest Path Update Algorithms}. Single source shortest path update algorithms have been studied in \cite{bauer2009batch,d2013dynamically,frigioni1996fully,narvaez2000new,mcquillan1980new}.
%In general, these algorithms preprocess the shortest paths starting from given source node, and store them as a shortest path tree(SPT). To update the SPT, the basic idea is to utilize the information of the outdated SPT and update only the part of the SPT that is affected by the changed edge. However, for the shortest paths in cache, they cannot be stored in a single shortest path tree. Alternatively, although we can generate multiple SPT to represent the shortest paths, we still need to perform the algorithm for each SPT, which is a expensive operation, especially there exist many SPT.
%}
%
%\modify{
%Besides, in order to accelerate the computation of shortest paths, some works developed the preprocessing techniques by computing additional data in a preprocessing phase. Arc-Flags is a typical approach proposed by \cite{lauther2004extremely} designed for static network. \cite{berrettini2009arc,d2014fully} study the problem of updating Arc-Flags in dynamic network.
%}
%\fi
%
%
%\iffalse
%With respect to the monitoring of shortest paths, \cite{lee2007} proposes an ellipse bound method (EBM), where each shortest path corresponds to an elliptic geographical area and all the updated edges are considered to affect their corresponding paths. The shortcoming of such a method is that they cover a large number of unrelated edges and a lot of computational time is wasted to re-compute unchanged paths. \cite{tian2009monitoring} develops the notion of query scope to identify paths invalidation and devise a partial path computation algorithm (PPCA) to quickly re-compute the updated paths. PPCA achieves superior performance in query processing time, however, it may lead more I/O cost due to extra network traversals, and it does not suit for the case that the network changes a lot. \cite{d2014fully} dynamically updates Arc-Flags to support speed-up techniques on changing networks, however, it needs to maintain a shortest path tree and the preprocessing of Arc-Flags is also very time-consuming.
%\fi

\modify{
\subsection{Caching Problem}
}
\modify{
Another steam is the caching problem. Caching techniques have been studied extensively in many domains \citep{o1993lru,markatos2001caching}. The main focus is to decide the contents that are cached.
}

\modify{
\textbf{Web Search Caching}.
In \cite{markatos2001caching}, caching strategies are classified into static caching and dynamic caching.
Static caching does not change the contents of cache once the cache is loaded based on the historical log \citep{altingovde2009cost,baeza2003three,baeza2007impact}. The static caching usually updates periodically based on the latest query log.
Dynamic caching focuses on suiting for new arrival queries and weeding out old contents \citep{gan2009improved,Long2005Three-Level,markatos2001caching}.
}

\modify{
\textbf{Shortest Path Caching}. \cite{thomsen2012effective,li2013,thomsen2014concise} have studied the caching of shortest paths. In details, \cite{thomsen2012effective}~is the pioneer work on caching of shortest paths. They proposed a cost-oriented model to quantify the benefit of loading paths to a cache. Based on this model, a static caching strategy is proposed to initialize the cache. \cite{li2013}~also focuses on a static strategy. In order to store as many paths as possible, an improved cost-oriented model was proposed to measure paths appeared in the query log. \cite{thomsen2014concise} proposed the notion of generically concise shortest paths that enables a trade-off between the size of paths and the number of queries answered by paths. They considered both the static and dynamic caching strategies for selecting generically concise paths. For the dynamic strategy, they update the cache in a least-recently-used policy when missing targets occur.
}

\modify{
In applications of caching problem such as web search caching, queries change with time while the data (answers) will change in no way. The inefficiency of cache is fully resulted from the change of queries. Hence, ``dynamic'' caching strategies are motivated by the change of queries.
However, in the shortest path caching problem, the change of graph (data) may also lead to the incorrectness of the answer and inefficiency of the cache.
In this paper, we investigate the problem of shortest path caching when the graph changes, and we extend the concept of ``dynamic'' caching to the change of data.
%None of the above works consider the caching strategy in the scenario of dynamic network, actually our proposed caching strategy can also be classified to dynamic caching strategy, since we update the cache as the change of graph. The main difference between our caching strategy and the conventional dynamic caching strategy is that, conventional dynamic caching usually utilizes the LRU policy the based on the changed queries, while our target is to refresh the affected shortest path in cache that caused by the change of graph.
}

\iffalse
\modify{
\textbf{Static caching}. For the static caching, through \cite{altingovde2009cost,baeza2003three,baeza2007impact}, an interesting finding is that the frequency of queries basically follows a Zipfian distribution, where a small portion of queries appear frequently in the query log. Moreover, these popular queries typically last for quite a long period of time. Therefore by exploiting a query log which records previous queries, static caching figures out the most frequent queries as the most popular ones, and loads them into cache.
\cite{thomsen2012effective} and \cite{li2013} carry out static caching techniques for caching shortest paths. They utilize statistics from query logs to estimate the benefit of caching a shortest path and employ greedy algorithms to load beneficial paths to caches. At the same time, the cache utilization is maximized.
However, both methods fully rely on the historical logs, making cached contents over fit the queries in historical logs. As we all know, logs and future queries only have similar trends in a time period, however, they cannot be totally the same.
}



\modify{
\textbf{Dynamic caching}. For the dynamic caching \citep{gan2009improved,Long2005Three-Level,markatos2001caching}, no work talks about how to refresh a shortest path cache dynamically, especially when the underlying graph changes. Refreshment of conventional caches has been studied, which concentrates on adaption to incoming queries.
Taking the Least-Recently-Used method as an example, whenever a cache fails to answer a request, the least recently used result in a cache is replaced by the currently queried result. With the help of its most up-to-date merit, this approach adapts quickly to new queries; but on the other hand, to maintain the cache up-to-date, this approach incurs overhead for frequently updating the cache.
}


\modify{
It is commonly known that HQF and LRU are not efficient enough for the shortest path caching.
\textbf{Hybrid caching}. Static Dynamic Cache (SDC) method is proposed in \cite{Fagni2006Boosting}. It is a new caching strategy which aims to efficiently exploit the temporal and spatial locality presented in the stream of processed queries. SDC extracts the results of the most frequently submitted queries from historical usage data and stores them in a static, read-only portion of the cache. The remaining entries of the cache are dynamically managed according to a given replacement policy and are used for those queries that cannot be satisfied by the static portion.
}
\fi
