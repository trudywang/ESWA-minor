\section{Preliminaries}
\label{sec:pre}
\begin{definition}
\label{def:Graph model}
{\textbf{Graph Model.}} \modify{ Public transport network are usually represented as directed graphs} $G$ = ($V$, $E$) where $V$ = \{$v_1$, $v_2$,..., $v_n$\} is the set of nodes and $E$ is the set of edges. Each edge in this graph is represented by a pair of nodes $e$($v_i$, $v_j$). Moreover, the weight of each edge $e(v_i$, $v_j$) is denoted as $w(v_i$, $v_j$).
\end{definition}

\begin{definition}
\label{def:Queries and Answers}
{\textbf{Shortest Path.}} For any given graph $G$ = ($V$, $E$), the shortest path from node $v_a$ to node $v_b$ is denoted as $P_{a,b}=\langle v_{x_0},v_{x_1},...,v_{x_n}\rangle$ here $v_{x_0}=v_a$ , $v_{x_n}=v_b$ and the distance is $D_{a,b}=\sum_{i=0}^{n-1}w(v_{x_i},v_{x_{i+1}})$.
\end{definition}

\modify{
\begin{definition}
\label{def:SP Query}
{\textbf{Shortest Path Query.}} For any given graph $G$ = ($V$, $E$),  $Q_{a,b}$ is used to query the shortest path $P_{a,b}$ between nodes $v_a$ and $v_b$.
\end{definition}
}

\begin{definition}
\label{def:Cache}
{\textbf{Cache.}} A cache is denoted by $\Omega$ and its capacity is $|\Omega|$. $\Psi$ refers to the caching content. The size of a cache and caching content are measured in terms of number of nodes. It is clear that the size of $\Psi$ is always not larger than the cache capacity, i.e., $|\Psi|\leq |\Omega|$ all the time.
\end{definition}

\begin{definition}
\label{def:Subpaths set S(Pa,b)}
{\textbf{Subpaths Set $S(P_{a,b})$.}} All the subpaths of $P_{a,b}$ compose set $S(P_{a,b})$.
\end{definition}

\begin{equation}
\label{eq:subpaths}
S(P_{a,b})=\{P_{v_i,v_j}|v_i \in P_{a,b}~and~v_j \in P_{a,b},~for~any~i<j\} .
\end{equation}

For example, for the shortest path $P_{1,8}$ in Fig.~\ref{fig:decrease_example}, its subpaths set $S(P_{1,8})$ contains paths $P_{1,3}$, $P_{1,6}$, $P_{1,8}$, $P_{3,6}$, $P_{3,8}$ and $P_{6,8}$.

\modify{
\begin{definition}
{\textbf{Nodes Distance.}} For graph $G$, for any two connected nodes $v_i$ and $v_j$, their nodes distance is defined as the minimum number of edges between $v_i$ and $v_j$.
\end{definition}
}
\modify{
For instance, the node distance from $v_0$ to $v_6$ is 3, and $v_8$ to $v_6$ is 1 in Fig.~\ref{fig:decrease_example}.
}

\begin{definition}
\label{def:Affected path}
{\textbf{Affected Paths.}} Suppose the shortest path between $v_a$ and $v_b$ is originally $P_{a,b}=\langle v_a,v_i,...,v_j,v_b\rangle$. Due to the change of edge weights, if the shortest path between $v_a$ and $v_b$ becomes $P'_{a,b}=\langle v_a,v_m,...,v_n,v_b\rangle$ ($P'_{a,b} \neq P_{a,b}$), then $P_{a,b}$ is called an affected path.
\end{definition}

\begin{lemma}
\label{lemmma:Optiaml subpath property}
{\textbf{Optimal Subpath Property}} \cite{cormen2001introduction}. A subpath of any shortest path is a shortest path of the ends of that subpath. That is, for a given shortest path $P_{a,b}=\langle v_{a}, v_{i}, ...,$ $ v_{j}, ...,v_{k},..., v_{b}\rangle$ and any pair of nodes ($v_{i}$ ,$v_{j}$) on it, the subpath $\langle v_{i},...,v_{j}\rangle$ is the shortest path of $P_{i,j}$.
\end{lemma}

Lemma 1 tells us that if we store a certain shortest path in a cache, we can answer all of the shortest path queries whose end nodes are on that path.

\mbox{}

\noindent{\bf Problem description.}



